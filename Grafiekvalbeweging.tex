\documentclass[]{standalone}
\usepackage{tikz,pgfplots,filecontents,pgfplotstable,siunitx,mathpazo}
\begin{filecontents}{ValbewegingTable.csv}
0		1.77	-0.05 
0.03	1.76	-0.56 
0.06	1.73	-0.61 
0.10	1.72	-0.74 
0.13	1.68	-1.48 
0.16	1.62	-1.85 
0.20	1.56	-1.96 
0.23	1.49	-2.46 
0.26	1.40	-2.96 
0.30	1.29	-3.62 
0.33	1.15	-4.02 
0.36	1.02	-1.90 
0.40	1.03	-2.38 
0.43	0.86	-4.66 
0.46	0.72	-4.84 
0.50	0.54	-5.58 
0.53	0.34	-5.79
\end{filecontents}

\begin{document}
\begin{tikzpicture}
\begin{axis}[minor tick num=1,
			xlabel=$t$ (\si{\second}),
			ylabel=$y_f$ (\si{\meter})		
			]
			\addplot[restrict x to domain=0:.6,restrict y to domain=0:1.8,color=blue,samples=1000]{-4.905*x^2+1.8}; \addlegendentry{Theorie};
\addplot[only marks,
		 mark options={fill=gray,scale=1.0},
		 mark=square*
		 ] 
		 table[x index=0, y index=1] {ValbewegingTable.csv};
\addplot [black] gnuplot [raw gnuplot] { % "raw gnuplot" allows us to use arbitrary gnuplot commands
            f(x) = a*x**2+1.8; % Define the function to fit
            fit f(x) "ValbewegingTable.csv" u 1:2 via a; % Select the file, the columns (indexing starts at 1) and the variables
            plot[x=0:0.6] f(x) wl; % Specify the range to plot
    }; \addlegendentry{ Experiment};
\end{axis}
\end{tikzpicture}
\end{document}
